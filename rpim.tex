\documentclass[11pt]{book}

\usepackage{alltt}
\usepackage{amsfonts}
\usepackage{amsmath}
\usepackage{amssymb}
\usepackage{ccicons}

% Careful: mathbbol apparently doesn't like 11pt
% The stmryrd error about 11pt is actually caused by mathbbol
% See http://www.latex-community.org/forum/viewtopic.php?f=4&t=2099
% 10pt or 12pt are apparently okay, though.
%\usepackage[cspex,bbgreekl]{mathbbol}

\usepackage{slatex}
\usepackage{stmaryrd}
\usepackage{url}
\usepackage{verbatim}

\begin{document}
\begin{schemeregion}

\title{Relational Programming in miniKanren}
\author{William E. Byrd}
\maketitle
\tableofcontents

\newpage
\huge
\noindent
\ccLogo
\ccAttribution
%\ccby

\large
\noindent
This work is licensed under a Creative Commons Attribution 3.0 Unported License.
(CC BY 3.0)

\noindent
\url{http://creativecommons.org/licenses/by/3.0/}
\normalsize

\chapter{Acknowledgements}
\chapter{Preface}

\chapter{Introduction to Relational Programming}

\chapter{Introduction to miniKanren}
\section{Core miniKanren}
\section{Constraint Logic Programming}

\chapter{Translating from Scheme to miniKanren}
\subsection{A-Normal Form}
\subsection{Defunctionalization}
\subsection{Pattern Matching}

\chapter{Exploring the Chomsky Hierarchy}
\section{Regular Expression Matching}
\section{Deterministic Finite Automata}

\chapter{Relational Exploration of Programming Languages Fundamentals}
\section{lexical scope}
\section{Relational Program Transformations}
\subsection{Continuation-Passing Style}

\chapter{Relational Interpreters}
\section{Relational Scheme Interpreter}
\subsection{Generating Quines}
\section{Relational CESK Machine}

\chapter{Type Inference}
\section{Type Inhabitation}

\chapter{Probabilistic Logic Programming}

\chapter{Implementing miniKanren}
\section{Unification}
\section{An Embedding in Scheme}
\section{A miniKanren Interpreter}
\section{A Meta-circular miniKanren Interpreter}
\section{An Abstract Machine for miniKanren}
\section{A Relational miniKanren Interpreter}

\chapter{The Future of Relational Programming}
\section{Open Problems}

\end{schemeregion}
\end{document}
