\chapter{Introduction}

\begin{fullwidth}
\begin{flushright}
\Large
\textit{gl hf!}

\wspace

(Traditional greeting in the Koprulu Sector)
\normalsize
\end{flushright}
\end{fullwidth}


%%% I need to decide who my intended audience is, and always keep them
%%% in mind.  I need to be especially careful about referring to
%%% things PL folks and CS PhDs take for granted.  In the margin notes
%%% I can geek out all I want.


%%% Do I believe this?  Yes.  However, unless the reader knows what logic programming is, this claim won't mean much.

% Logic programming has failed.

% More specifically, logic programming, as currently practiced, has failed to live up to its promise as the ultimate in declarative programming.



%%% I guess this paragraph is okay, as far as it goes.  But once
%%% again, mind my audience.  Also, concentrate on ideas over specific
%%% languages or implementations whenever possible.  Want the book to
%%% be useful and interesting even for readers who have no interest in
%%% ever using miniKanren.

% This book describes the miniKanren programming language\marginnote{miniKanren is actually a family of related programming languages, embedded in a variety of host languages. Unless otherwise specifified, we use the term ``miniKanren'' to refer to the entire family of languages, including {\tt core.logic}, cKanren, and any other variants.}, and how to use miniKanren to write programs in a {\em relational} style.  Just as functional programming is based on the notion of mathematical functions, relational programming is based on the notion of mathematical relations.  For this reason, relational programs are in some ways similar to relational databases.\marginnote{Indeed, SQL is arguably the most successful relational (and declarative) programming language.  As we will see, miniKanren is more expressive than SQL, although this increased expressivity has disadvantages.}  Relational programming goes beyond relatioal databases, however, and can be seen as an especially pure variant of logic programming.  Although both views of relational programming are valid and useful, we shall concentrate on an alternate, complementary view of relational programming, based on the transformation of functions in a functional programming language to relations in miniKanren.



% the promise of declarative programming

% advantages of relational programming

% limitations of relational programming

% using miniKanren for non-relational programming
