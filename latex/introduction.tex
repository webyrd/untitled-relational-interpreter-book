\chapter{Introduction}

\vspace{-1.5cm}
\begin{fullwidth}
\begin{flushright}
\Large
\textit{gl hf!}

\wspace

---Greg ``IdrA'' Fields
%(Traditional greeting in the Koprulu Sector)
\normalsize
\end{flushright}
\end{fullwidth}

Relational programs generalize functional programs, in that they do
not distinguish between the ``input'' arguments passed to a function
and the ``output'' result returned by that function.
%
For example, consider a two-argument variant of Scheme's addition function, restricted to natural\marginnote{The \emph{natural numbers} are the non-negative integers: 0, 1, $\ldots$} numbers: \mbox{\scheme|(+ 3 4) => 7|}.
%
A relational version of addition, \scheme|+o|, takes three arguments:\marginnote{Here we are taking a notational liberty, as \scheme|+o| expects 3 and 4 to be represented as \emph{binary}, \emph{little-endian} lists: \mbox{\scheme|(1 1)|} and \mbox{\scheme|(0 0 1)|}, respectively. Zero is uniquely represented as the empty list, \scheme|()|.  To ensure a unique representation of each number, lists may not end with the digit \schemeresult|0|.  This numeric representation is extremely flexible, since the lists can contain logic variables---for example, the list \mbox{\scheme|`(1 . ,x)|} represents any odd natural number, while \mbox{\scheme|`(0 . ,x)|} represents any positive even natural. \\ We can also perform relational arithmetic on built-in Scheme numbers, using \emph{Constraint Logic Programming over Finite Domains}, or \emph{CLP(FD)}; as we will see, CLP(FD) is faster, but less general, than \scheme|+o| and friends (\scheme|*o|, \scheme|/o|, etc.).} \mbox{\scheme|(+o 3 4 z)|},
where \scheme|z| is a \emph{logic variable} representing the result
of adding the first two arguments of \scheme|+o|. In this case
\scheme|z| is associated with \schemeresult|7|.
%
More interestingly, we can write \mbox{\scheme|(+o 3 y 7)|}, which
associates \scheme|y| with \schemeresult|4|; our addition relation
also performs subtraction.
%
We can also write \mbox{\scheme|(+o x y 7)|}, which associates
\scheme|x| and \scheme|y| with all pairs of natural numbers that sum
to \schemeresult|7|; \scheme|+o| produces multiple
answers, including \mbox{\scheme|x = 3|} and \mbox{\scheme|y = 4|}, and
\mbox{\scheme|x = 0|} and \mbox{\scheme|y = 7|}.
%
Finally, we can write \mbox{\scheme|(+o x y z)|}, which enumerates 
all triples of natural numbers $(x, y, z)$ such that $x + y = z$; 
 \scheme|+o| produces infinitely many answers.
%
Informally, we say that the call \mbox{\scheme|(+o 3 4 z)|} runs the \scheme|+o| relation ``forwards,''
%(that is, in the same direction as the corresponding Scheme function),
while the calls \mbox{\scheme|(+o 3 y 7)|}, \mbox{\scheme|(+o x y 7)|}, and \mbox{\scheme|(+o x y z)|} run ``backwards.''
%Informally, we say that the \scheme|+o| relation can ``run backwards,'' in
%contrast to Scheme's \scheme|+| function, which only can be used in the
%``forwards'' direction.

We will see that the remarkable flexibility of the \scheme|+o|
relation is exhibited by more complex relations, such as interpreters,
type inferencers, and finite-state machines.
%
For example, we will write a relational interpreter for a subset
of Scheme, \scheme|evalo|. Running \scheme|evalo| forwards, the call\marginnote{As might be expected, \scheme|evalo| will interpret the quoted list \mbox{\schemeresult|(+ x 2)|} as the call \mbox{\scheme|(+o x 2 val)|}, where the logic variable \scheme|x| is associated with \schemeresult|4|.}
\mbox{\scheme|(evalo '((lambda (x) (+ x 2)) 4) val)|} associates
\scheme|val| with \schemeresult|6|. Running 
backwards is more interesting: \mbox{\scheme|(evalo exp '6)|} generates legal
Scheme expressions that \emph{evaluate} to 6, while
\mbox{\scheme|(evalo exp exp)|} generates \emph{quines}, which are Scheme expressions that evaluate to themselves.

%\marginnote{Douglas Hofstadter coined the term \emph{quine}, in honor of logician Willard Van Orman Quine (1908--2000).  Writing quines has long been a favorite hacker activity, and quines are often featured in the The International Obfuscated C Code Contest (\url{http://www.ioccc.org/}). A delightful introduction to quines can be found in Doug's classic, \emph{GEB}: \cite{GEB79}}

%This book will teach you how to write relations that produce
%interesting answers when running forwards and backwards.
%
%Running a relation in the forward direction is usually straight-forward.
%
%Running a relation backwards is where the fun begins.\marginnote{All
%too often this fun never ends, as the relation diverges (loops forever)!}
%
%I hope you'll enjoy the neverending fun of relational programming as
%much as I have.



%%% I need to decide who my intended audience is, and always keep them
%%% in mind.  I need to be especially careful about referring to
%%% things PL folks and CS PhDs take for granted.  In the margin notes
%%% I can geek out all I want.


%%% Do I believe this?  Yes.  However, unless the reader knows what logic programming is, this claim won't mean much.

% Logic programming has failed.

% More specifically, logic programming, as currently practiced, has failed to live up to its promise as the ultimate in declarative programming.



%%% I guess this paragraph is okay, as far as it goes.  But once
%%% again, mind my audience.  Also, concentrate on ideas over specific
%%% languages or implementations whenever possible.  Want the book to
%%% be useful and interesting even for readers who have no interest in
%%% ever using miniKanren.

% This book describes the miniKanren programming language\marginnote{miniKanren is actually a family of related programming languages, embedded in a variety of host languages. Unless otherwise specifified, we use the term ``miniKanren'' to refer to the entire family of languages, including {\tt core.logic}, cKanren, and any other variants.}, and how to use miniKanren to write programs in a {\em relational} style.  Just as functional programming is based on the notion of mathematical functions, relational programming is based on the notion of mathematical relations.  For this reason, relational programs are in some ways similar to relational databases.\marginnote{Indeed, SQL is arguably the most successful relational (and declarative) programming language.  As we will see, miniKanren is more expressive than SQL, although this increased expressivity has disadvantages.}  Relational programming goes beyond relatioal databases, however, and can be seen as an especially pure variant of logic programming.  Although both views of relational programming are valid and useful, we shall concentrate on an alternate, complementary view of relational programming, based on the transformation of functions in a functional programming language to relations in miniKanren.



% the promise of declarative programming

% advantages of relational programming

% limitations of relational programming

% using miniKanren for non-relational programming
